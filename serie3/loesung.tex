\documentclass[a4paper,abstracton]{scrartcl}
\usepackage[ngerman]{babel} %deutsch
\usepackage[utf8]{inputenc} %Umlaute
\usepackage{amsmath} %math
\usepackage{amsfonts} %math
\usepackage{amssymb} %math
\usepackage{xcolor,graphicx} %grafike inelade
\usepackage{fancyhdr}
\usepackage{float}
\usepackage{units}
\usepackage{textcomp}
\usepackage{amstext}
\usepackage{graphicx}
\usepackage[pdftex]{hyperref}  %verlinkige, immer zungerscht!
 \usepackage{listings}
  \usepackage{courier}
\usepackage{color}

 \lstset{
         basicstyle=\footnotesize\ttfamily\color{black}, % Standardschrift
				 numbers=left,               % Ort der Zeilennummern
         numberstyle=\tiny,          % Stil der Zeilennummern
         %stepnumber=2,               % Abstand zwischen den Zeilennummern
         %numbersep=5pt,              % Abstand der Nummern zum Text
         tabsize=2,                  % Groesse von Tabs
         extendedchars=true,         %
         breaklines=true,            % Zeilen werden Umgebrochen        
         keywordstyle=\textbf,    % Stil der Keywords
         keywordstyle=[2]\textbf,    %
         keywordstyle=[3]\textbf,    %
				 keywordstyle=[4]\textbf,    %
         stringstyle=\ttfamily, % Farbe der String
         showspaces=false,           % Leerzeichen anzeigen ?
         showtabs=false,             % Tabs anzeigen ?
      %   xleftmargin=17pt,
     %    framexleftmargin=17pt,
       %  framexrightmargin=5pt,
      %   framexbottommargin=4pt,
        % backgroundcolor=\color{lightgray},
         showstringspaces=false,      % Leerzeichen in Strings anzeigen ?       
 }
 \lstloadlanguages{% Check Dokumentation for further languages ...
         %[Visual]Basic
         %Pascal
         C
         %C++
         %XML
         %HTML
         %Java
 }    %\DeclareCaptionFont{blue}{\color{blue}} 

  %\captionsetup[lstlisting]{singlelinecheck=false, labelfont={blue}, textfont={blue}}
  \usepackage{caption}
\DeclareCaptionFont{white}{\color{white}}
\DeclareCaptionFormat{listing}{\colorbox[cmyk]{0.43, 0.35, 0.35,0.01}{\parbox{\textwidth}{\hspace{15pt}#1#2#3}}}
\captionsetup[lstlisting]{format=listing,labelfont=white,textfont=white, singlelinecheck=false, margin=0pt, font={bf,footnotesize}} % sourcecode formatierung

\providecommand{\tabularnewline}{\\}

\title{Rechnerarchitektur Serie 3}
\author{Dominik Bodenmann 08-103-053\\
	Orlando Signer 12-119-715\\}


\begin{document}
\maketitle

\section{Theorie-Teil}
\subsection{Aufgabe 1}
$
\begin{array}{l|c|c|c|c||c|c||c|c|}
 & t & Freq & CPI & Freq*CPI & CPI & Freq*CPI & CPI & Freq*CPI\\
\hline
ALU & 5nsec & 25\% & 2 & 0.5 & 2 & 0.5 & {\bf 1} & 0.25\\
LOAD & 10nsec & 25\% & 4 & 1.0 & {\bf 6} & 1.5 & 4 & 1.0\\
STORE & 7.5nsec & 25\% & 3 & 0.75 & 3 & 0.75 & 1 & 0.75\\
Branch & 7.5nsec & 25\% & 3 & 0.75 & 3 & 0.75 & 1 & 0.75\\
\hline
 & & & & 3.0 & & 3.5 & & 2.75
\hline\end{array}
$
\\
\begin{enumerate}
	\item Eine Maschine, die f"ur die LOAD Instruktion 6 Taktzyklen braucht, ist also\\$ 3.5/3.0 - 1 = 16.7\% $ langsamer.
	\item Eine CPU, bei der die ALU doppelt so schnell arbeitet, ist also\\$ 2.75/3.0 -1 = 8.3\% $ schneller.
\end{enumerate}

\subsection{Aufgabe 2}
\begin{enumerate}
	\item Darauf kann die R"ucksprungadresse f"ur die Vortsetzung der Programmbearbeitung gespeichert werden.
	\item Darauf k"onnen die Aufrufparameter gelegt werden, damit sie von der Subroutine gelesen werden k"onnen.
\end{enumerate}

\subsection{Aufgabe 3}
Damit der Overflow behandelt werden kann: \\\\
$
\begin{array}{|c|c|c|c|c|c|c|}
Vorzeichen A & Vorzeichen B & B invert & Carry out & Vorz. Result. & Korr. Vorz. Res. & Overflow\\
\hline
0&0&0&0&0&0&0\\
0&0&1&0&1&0&1\\
0&1&0&0&1&1&0\\
0&1&1&1&0&0&0\\
1&0&0&0&1&1&0\\
1&0&1&1&0&0&0\\
1&1&0&1&0&1&1\\
1&1&1&1&1&1&0\\
\hline\end{array}
$ \\\\
Die ersten 5 Argumente werden f"ur die Overflowdetection gebraucht. 
Ist das B invert Bit nicht gleich dem Carry out Bit, so gibt es einen Overflow. (Bei ALU31, da da die Vorzeichen abgespeichert sind)

\subsection{Aufgabe 4}
\begin{itemize}
	\item Beim slt-Befehl wird a-b gerechnet. Falls $ a-b < 0 $ erh"alt man im Vorzeichenbit (ALU31) eine 1.
	\item Die ALU unterst"utzt diesen Befehl, indem sie das Set des ALU31 Bits mit dem Less des ALU0 Bits verbindet. 
	Alle anderen Less sind 0;
\end{itemize}

\subsection{Aufgabe 5}
\begin{lstlisting}[caption=MIPS push und pop]
# Push und pop auf dem Stack. $sp bezeichnet den Stackpointer.
push:
subi $sp $sp 4
sw  $t0 0($sp)

pop:
lw $t0 0($sp)
addi $sp $sp 4
\end{lstlisting}
$

\subsection{Aufgabe 6}

\subsection{Aufgabe 7}
$\begin{array}{}
Befehl & A invert & B invert & Operation\\
\hline
and &0&0&0&0\\
or &0&0&0&1\\
add &0&0&1&0\\
subtract &0&1&1&0\\
slt &0&1&1&1\\
nor &1&1&0&0\\
\hline
\end{array}$\\
\begin{itemize}
	\item[and] Mit dem Opcode 00 wird der erste Eingand des Operationsmuxes (das AND-Gatter) angesteuert, 
	und somit muss A und B nicht invertiert werden.
	\item[or] Gleich wie oben. Mit dem Opcode 01 kann direkt der zweite Eingang des Operationsmuxes (das OR-Gatter) 
	angesteuert werden, und somit m"ussen A und B wiederum nicht invertiert werden.
	\item[add] Wiederum kann mit dem Opcode 10 der dritte Eingang des Operationsmuxes (Addition) angesteuert werden, 
	und Aund B m"ussen wiederum nicht invertiert werden.
	\item[subtract] Wenn man b negiert (also invertiert) kann man einfach noch eine Addition ausf"uhren (Opcode 10)
	\item[slt] Es wird wieder $a-b$ gerechnet (siehe subtract). Also B invertieren. 
	Falls B gr"osser als A ist, soll das 31ste Bit den Wert 1 haben, welcher ebenfalls aufs erste Bit zur"uckgef"uhrt wird. 
	Damit kann man beim Operationsmux den vierten Eingang (Opcode 11) w"ahlen, und erh"alt nun als Ausgabe eine 1, 
	falls B gr"osser als A ist und sonst eine 0.
	\item[nor] Da $ a NOR b = not a AND not b $ ist, kann man A und B invertieren und das AND-Gatter (Opcode 00) anlegen.
\end{itemize}

\newpage

\section{Programmierteil}
\lstinputlisting[language=C, caption=compile.c]{src/compile.c}
\newpage
\lstinputlisting[language=C, caption=mips.c]{src/mips.c}
\newpage
\lstinputlisting[language=C, caption=memory.c]{src/memory.c}

\end{document}
