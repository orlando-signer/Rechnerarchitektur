\documentclass[a4paper,abstracton]{scrartcl}
\usepackage[ngerman]{babel} %deutsch
\usepackage[utf8]{inputenc} %Umlaute
\usepackage{amsmath} %math
\usepackage{amsfonts} %math
\usepackage{amssymb} %math
\usepackage{xcolor,graphicx} %grafike inelade
\usepackage{fancyhdr}
\usepackage{float}
\usepackage{units}
\usepackage{textcomp}
\usepackage{amstext}
\usepackage{graphicx}
\usepackage[pdftex]{hyperref}  %verlinkige, immer zungerscht!
 \usepackage{listings}
  \usepackage{courier}
\usepackage{color}

 \lstset{
         basicstyle=\footnotesize\ttfamily\color{black}, % Standardschrift
				 numbers=left,               % Ort der Zeilennummern
         numberstyle=\tiny,          % Stil der Zeilennummern
         %stepnumber=2,               % Abstand zwischen den Zeilennummern
         %numbersep=5pt,              % Abstand der Nummern zum Text
         tabsize=2,                  % Groesse von Tabs
         extendedchars=true,         %
         breaklines=true,            % Zeilen werden Umgebrochen        
         keywordstyle=\textbf,    % Stil der Keywords
         keywordstyle=[2]\textbf,    %
         keywordstyle=[3]\textbf,    %
				 keywordstyle=[4]\textbf,    %
         stringstyle=\ttfamily, % Farbe der String
         showspaces=false,           % Leerzeichen anzeigen ?
         showtabs=false,             % Tabs anzeigen ?
      %   xleftmargin=17pt,
     %    framexleftmargin=17pt,
       %  framexrightmargin=5pt,
      %   framexbottommargin=4pt,
        % backgroundcolor=\color{lightgray},
         showstringspaces=false,      % Leerzeichen in Strings anzeigen ?       
 }
 \lstloadlanguages{% Check Dokumentation for further languages ...
         %[Visual]Basic
         %Pascal
         C
         %C++
         %XML
         %HTML
         %Java
 }    %\DeclareCaptionFont{blue}{\color{blue}} 

  %\captionsetup[lstlisting]{singlelinecheck=false, labelfont={blue}, textfont={blue}}
  \usepackage{caption}
\DeclareCaptionFont{white}{\color{white}}
\DeclareCaptionFormat{listing}{\colorbox[cmyk]{0.43, 0.35, 0.35,0.01}{\parbox{\textwidth}{\hspace{15pt}#1#2#3}}}
\captionsetup[lstlisting]{format=listing,labelfont=white,textfont=white, singlelinecheck=false, margin=0pt, font={bf,footnotesize}} % sourcecode formatierung

\providecommand{\tabularnewline}{\\}

\title{Rechnerarchitektur Serie 3}
\author{Dominik Bodenmann 08-103-053\\
	Orlando Signer 12-119-715\\}


\begin{document}
\maketitle

\section{Theorie-Teil}
\subsection{Aufgabe 1}
\begin{array}{l|c|c|c|c||c|c||c|c|}
 & t & Freq & CPI & Freq*CPI & CPI & Freq*CPI & CPI & Freq*CPI
\hline
ALU & 5nsec & 25\% & 2 & 0.5 & 2 & 0.5 & {\bf 1} & 0.25\\
LOAD & 10nsec & 25\% & 4 & 1.0 & {\bf 6} & 1.5 & 4 & 1.0\\
STORE & 7.5nsec & 25\% & 3 & 0.75 & 3 & 0.75 & 1 & 0.75\\
Branch & 7.5nsec & 25\% & 3 & 0.75 & 3 & 0.75 & 1 & 0.75\\
\hline
 & & & & 3.0 & & 3.5 & & 2.75
 \end{array}
Eine Maschine, die f"ur die LOAD Instruktion 6 Taktzyklen braucht, ist also $ 3.5/3.0 - 1 = 16.7\% $ langsamer.\\
Eine CPU, bei der die ALU doppelt so schnell arbeitet, ist also $ 2.75/3.0 -1 = 8.3\% $ schneller.

\subsection{Aufgabe 2}
\begin{enumerate}
\item Darauf kann die R"ucksprungadresse f"ur die Vortsetzung der Programmbearbeitung gespeichert werden.
\item Darauf k"onnen die Aufrufparameter gelegt werden, damit sie von der Subroutine gelesen werden k"onnen.

\subsection{Aufgabe 3}

\subsection{Aufgabe 4}

\subsection{Aufgabe 5}

\subsection{Aufgabe 6}

\subsection{Aufgabe 7}


\end{document}
