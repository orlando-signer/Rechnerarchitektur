\documentclass[a4paper,abstracton]{scrartcl}
\usepackage[ngerman]{babel} %deutsch
\usepackage[utf8]{inputenc} %Umlaute
\usepackage{amsmath} %math
\usepackage{amsfonts} %math
\usepackage{amssymb} %math
\usepackage{xcolor,graphicx} %grafike inelade
\usepackage{fancyhdr}
\usepackage{float}
\usepackage{units}
\usepackage{textcomp}
\usepackage{amstext}
\usepackage{graphicx}
\usepackage[pdftex]{hyperref}  %verlinkige, immer zungerscht!
\input{../tex/source_format} % sourcecode formatierung

\providecommand{\tabularnewline}{\\}

\title{Rechnerarchitektur Serie 2}
\author{Dominik Bodenmann 08-103-053\\
	Orlando Signer 12-119-715\\}


\begin{document}
\maketitle

\section{Theorie-Teil}
\subsection{Aufgabe 1}
\begin{lstlisting}[caption=Ausgabe]
A: 10
B: 11
C: 12
\end{lstlisting}

\subsection{Aufgabe 2}
\begin{lstlisting}[caption=Eigenschaften]
int * a: initialisiert in a einen Pointer auf einen Integer
int const * b: initialisiert in b einen Pointer auf einen konstanten Integer
int * const c: initialisiert in c einen konstanten Pointer auf einen Integer
int const * const d: initialisiert in d einen konstanten Pointer auf einen konstanten Integer
\end{lstlisting}

\subsection{Aufgabe 3}
\begin{lstlisting}[caption=Problem]
\end {lstlisting}
Es sollte nur bis i=9 gehen (Indizes von 0 bis 9), d.h. es gibt einen Segmentation Fault.

\subsection{Aufgabe 4}
\begin{lstlisting}[caption=Ausgabe]

\end{lstlisting}


\section{Programmierteil}
\lstinputlisting[language=C, caption=mips.c]{src/mips.c}
\newpage
\lstinputlisting[language=C, caption=test.c]{src/test.c}

\end{document}
