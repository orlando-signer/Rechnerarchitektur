\documentclass[a4paper,abstracton]{scrartcl}
\usepackage[ngerman]{babel} %deutsch
\usepackage[utf8]{inputenc} %Umlaute
\usepackage{amsmath} %math
\usepackage{amsfonts} %math
\usepackage{amssymb} %math
\usepackage{xcolor,graphicx} %grafike inelade
\usepackage{fancyhdr}
\usepackage{float}
\usepackage{units}
\usepackage{textcomp}
\usepackage{amstext}
\usepackage{graphicx}
\usepackage[pdftex]{hyperref}  %verlinkige, immer zungerscht!
 \usepackage{listings}
  \usepackage{courier}
\usepackage{color}

 \lstset{
         basicstyle=\footnotesize\ttfamily\color{black}, % Standardschrift
				 numbers=left,               % Ort der Zeilennummern
         numberstyle=\tiny,          % Stil der Zeilennummern
         %stepnumber=2,               % Abstand zwischen den Zeilennummern
         %numbersep=5pt,              % Abstand der Nummern zum Text
         tabsize=2,                  % Groesse von Tabs
         extendedchars=true,         %
         breaklines=true,            % Zeilen werden Umgebrochen        
         keywordstyle=\textbf,    % Stil der Keywords
         keywordstyle=[2]\textbf,    %
         keywordstyle=[3]\textbf,    %
				 keywordstyle=[4]\textbf,    %
         stringstyle=\ttfamily, % Farbe der String
         showspaces=false,           % Leerzeichen anzeigen ?
         showtabs=false,             % Tabs anzeigen ?
      %   xleftmargin=17pt,
     %    framexleftmargin=17pt,
       %  framexrightmargin=5pt,
      %   framexbottommargin=4pt,
        % backgroundcolor=\color{lightgray},
         showstringspaces=false,      % Leerzeichen in Strings anzeigen ?       
 }
 \lstloadlanguages{% Check Dokumentation for further languages ...
         %[Visual]Basic
         %Pascal
         C
         %C++
         %XML
         %HTML
         %Java
 }    %\DeclareCaptionFont{blue}{\color{blue}} 

  %\captionsetup[lstlisting]{singlelinecheck=false, labelfont={blue}, textfont={blue}}
  \usepackage{caption}
\DeclareCaptionFont{white}{\color{white}}
\DeclareCaptionFormat{listing}{\colorbox[cmyk]{0.43, 0.35, 0.35,0.01}{\parbox{\textwidth}{\hspace{15pt}#1#2#3}}}
\captionsetup[lstlisting]{format=listing,labelfont=white,textfont=white, singlelinecheck=false, margin=0pt, font={bf,footnotesize}} % sourcecode formatierung

\providecommand{\tabularnewline}{\\}

\title{Rechnerarchitektur Serie 2}
\author{Dominik Bodenmann 08-103-053\\
	Orlando Signer 12-119-715\\}


\begin{document}
\maketitle

\section{Theorie-Teil}
\subsection{Aufgabe 1}
\begin{lstlisting}[caption=Ausgabe]
A: 10
B: 11
C: 12
\end{lstlisting}

\subsection{Aufgabe 2}
\begin{lstlisting}[caption=Eigenschaften]
int * a: initialisiert in a einen Pointer auf einen Integer
int const * b: initialisiert in b einen Pointer auf einen konstanten Integer
int * const c: initialisiert in c einen konstanten Pointer auf einen Integer
int const * const d: initialisiert in d einen konstanten Pointer auf einen konstanten Integer
\end{lstlisting}

\subsection{Aufgabe 3}
\begin{lstlisting}[caption=Problem]
Es sollte nur bis i=9 gehen (Indizes von 0 bis 9), d.h. es gibt einen Segmentation Fault.
\end {lstlisting}

\subsection{Aufgabe 4}
\begin{lstlisting}[caption=C-Code]
while (\($s_2 != $zero\)){
	do something;
	\($s_2 -= $s_1;\)
}
\end{lstlisting}

\subsection{Aufgabe 5}
\begin{lstlisting}[caption=Erweiterung]
\(bge $s_2, $s_1, Label\) wird zu:
\(slt $at, $s_2, $s_1\)
\(beq $at, $zero, Label\)
\end{lstlisting}

\subsection{Aufgabe 6}
\begin{lstlisting}[caption=Dezimalwert]
Big Endian (Start bei der Adresse 10004):
\((0110 1101 1000 1100 0010 0100 0000 0000)_2 = (1837900800)_10\)
Little Endian (Start bei der Adresse 10007):
\((0000 0000 0010 0100 10000 1100 01110 1101)_2 = (2395245)_10\)
\end{lstlisting}

\subsection{Aufgabe 7}
\begin{lstlisting}[caption=Assembler]
\(lw $t_3, $t_1(#t_2)\)
\(mult $t_3, $t_0\)
\(sw #LO, $t_1($t_2)\)
\end{lstlisting}

\subsection{Aufgabe 8}
\begin{lstlisting}[caption=Laden in Register]
\(lw $s_2 9($s_1)\)
\end{lstlisting}

\section{Programmierteil}
\lstinputlisting[language=C, caption=mips.c]{src/mips.c}
\newpage
\lstinputlisting[language=C, caption=test.c]{src/test.c}

\end{document}
